%------------------ Schrifttyp in der Kopf- und Fußzeilen
\setkomafont{pageheadfoot}{\footnotesize\sffamily}

% Ändern der Abbildungs- und Tabellenbezeichnung (Niedermair S.157)
\addto\captionsngerman{\renewcommand\figurename{Abb.}}
\addto\captionsngerman{\renewcommand\tablename{Tab.}}
\renewcommand\listfigurename{Abbildungen}

% Betrag eines Wertes
\newcommand{\abs}[1]{\lvert #1 \rvert} 

% Absatz mit Formatierung
\newcommand{\absatz}[1]{\textbf{\textsc{#1}}} 

% Anhang Verweis
\newcommand{\anhang}[1]{Anhang \vref{#1}}

% Aufgabe Zähler
\newcommand{\aufgabe}{\stepcounter{plus} Aufgabe \arabic{plus}}

% Itemize Abkürzungen
\newcommand{\bi}{\begin{itemize}}
\newcommand{\ei}{\end{itemize}}

% Compact Enumerate Abkürzungen
\newcommand{\bce}{\begin{compactenum}}
\newcommand{\ece}{\end{compactenum}}

% Begin und End Equation
\newcommand{\be}{\begin{equation}}
\newcommand{\ee}{\end{equation}}

% Realteil einer Zahl
\newcommand{\real}[1]{\text{Re}\left\{#1\right\}}

% Multicolumn für Tabellen
\newcommand{\mc}{\multicolumn}

% Paralleltext
\newcommand{\pl}[1]{\ParallelLText{#1}}
\newcommand{\pr}[1]{\ParallelRText{#1}}
\newcommand{\pp}{\ParallelPar}

% Text in Rot
\newcommand{\textrot}[1]{\textcolor{red}{#1}}

% Abkürzung für multicolumn
\newcommand{\tab}[2]{\multicolumn{1}{#1}{#2}}

% Unterstreichen
\newcommand{\ul}[1]{\underline{#1}}

% Vspace Abkürzung
\newcommand{\vsf}{\vspace{5pt}}

% Abkürzungen für häufig verwendete Begriffe
\newcommand{\zb}{z.B.\,}
\newcommand{\idr}{i.d.R.\,}

% Zähler für Beispiele
\newcounter{req}
\newcommand{\zaehler}[1]{\refstepcounter{req}{#1} \thereq}





