% VERSICHERUNG
% ---------------------------------------------------------
\chapter*{Versicherung}
Hiermit versichere ich, dass ich die vorliegende Arbeit selbständig und ohne unzulässige Hilfe Dritter verfasst haben. Die aus fremden Quellen direkt oder indirekt übernommenen Texte, Gedankengänge, Konzepte usw. in meinen Ausführungen habe ich als solche eindeutig gekennzeichnet und mit vollständigen Verweisen auf die jeweilige Urheberschaft und Quelle versehen.\\ \\ 
Zudem kam der Generative Pre-trained Transformer ChatGPT zum Einsatz, um die eigenständig recherchierten Informationen in einen klar verständlichen Text umzuformulieren. Verbesserungen am Text durch ChatGPT wurden nicht gesondert gekennzeichnet, wohingegen durch ChatGPT nach präziser Anweisung generierte Textpassagen stets als solche markiert wurden.\\ \\
Mir ist bekannt, dass ein Täuschungsversuch vorliegt, wenn sich eine der vorstehenden Versicherungen als unrichtig erweist.\\ \\
\\
Wiesbaden, {\today} 
\\ 
\\
Ort, Datum und Unterschrift Verfasser
\newpage
% Gruppenarbeiten
% Hiermit versichern wir, dass wir die vorliegende Arbeit selbständig und ohne unzulässige Hilfe Dritter verfasst haben. Die aus fremden Quellen direkt oder indirekt übernommenen Texte, Gedankengänge, Konzepte usw. in unseren Ausführungen haben wir als solche eindeutig gekennzeichnet und mit vollständigen Verweisen auf die jeweilige Urheberschaft und Quelle versehen.\ \ Zudem kam der Generative Pre-trained Transformer ChatGPT zum Einsatz, um die eigenständig recherchierten Informationen in einen klar verständlichen Text umzuformulieren. Verbesserungen am Text durch ChatGPT wurden nicht gesondert gekennzeichnet, wohingegen durch ChatGPT nach präziser Anweisung generierte Textpassagen stets als solche markiert wurden.\ \ Uns ist bekannt, dass ein Täuschungsversuch vorliegt, wenn sich eine der vorstehenden Versicherungen als unrichtig erweist.\ \

% ABSTRACT
% ---------------------------------------------------------
\chapter*{Abstract}
Die vorliegende Arbeit untersucht die Optimierung von Mikrofluidiksystemen zur Anwendung in der medizinischen Diagnostik. Ziel der Arbeit ist es, die Effizienz des Probentransports in mikrofluidischen Chips zu verbessern, indem verschiedene Geometrien und Oberflächenbehandlungen verglichen werden. Zu diesem Zweck wurden sowohl numerische Simulationen als auch Laborversuche durchgeführt. Die Ergebnisse zeigen, dass eine spezielle Modifikation der Kanäle zu einer deutlichen Reduktion des Flüssigkeitswiderstandes und damit zu einer schnelleren Probentransportzeit führt. Diese Erkenntnisse tragen zur Weiterentwicklung effizienter und kostengünstiger Lab-on-a-Chip-Technologien bei, die für die schnelle medizinische Diagnostik von großem Interesse sind.
% Der Abstract ist eine kompakte Zusammenfassung der gesamten Arbeit. Er ist oft das Erste, was Leser oder Gutachter sehen, und sollte ihnen eine klare Vorstellung vom Inhalt und den Ergebnissen der Arbeit geben. Ein Abstract ist in der Regel zwischen 150-300 Wörtern lang.
% Inhalt des Abstract
% - Problemstellung: Was ist die zentrale Fragestellung oder das Problem, das die Arbeit adressiert?
% - Zielsetzung: Was soll erreicht werden?
% - Methoden: Welche Methoden wurden angewandt?
% - Ergebnisse: Was sind die wichtigsten Erkenntnisse oder Ergebnisse?
% - Schlussfolgerung: Welche Bedeutung haben die Ergebnisse?
\newpage


% ACKNOWLEDGEMENTS
% ---------------------------------------------------------
\chapter*{Danksagung}
Ich möchte mich herzlich bei meinem Betreuer Prof. Dr. Müller für die kompetente Betreuung und seine wertvollen Anregungen während der Erstellung dieser Arbeit bedanken. Ebenso danke ich meinen Kommilitonen, insbesondere Anna und Peter, für die zahlreichen konstruktiven Diskussionen und ihr hilfreiches Feedback. \\
Ein besonderer Dank gilt meiner Familie, die mich während des gesamten Studiums stets unterstützt hat und mir den Rücken freigehalten hat. Ohne ihre Ermutigung wäre diese Arbeit in dieser Form nicht möglich gewesen. \\
Zudem danke ich der Laborgruppe der XYZ Universität für die Bereitstellung der Laborinfrastruktur, die für die Durchführung der Experimente erforderlich war. \\
% Die Danksagung ist optional, wird aber häufig verwendet, um den Personen, Institutionen oder Organisationen zu danken, die die Arbeit unterstützt haben. Sie ist oft emotionaler als der Rest der Arbeit und dient dazu, die persönlichen Verbindungen und Unterstützungen zu würdigen, die zur erfolgreichen Durchführung der Arbeit beigetragen haben.
% Inhalt der Danksagung
% - Betreuer und wissenschaftliche Unterstützung: Dank an die Betreuer oder Dozenten, die fachlich unterstützt haben.
% - Kollegen und Mitstudierende: Dank an Kollegen, die vielleicht durch Feedback, Diskussionen oder Zusammenarbeit geholfen haben.
% - Familie und Freunde: Dank an die Familie und Freunde, die emotionale Unterstützung und Motivation gegeben haben.
% - Institutionen oder Förderer: Dank an Institutionen oder Programme, die eine finanzielle oder materielle Unterstützung bereitgestellt haben.
\newpage
